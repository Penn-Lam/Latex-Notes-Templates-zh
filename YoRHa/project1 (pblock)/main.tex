\documentclass[twocolumn]{article}

% Imports the catppuccin theme, using the mocha flavor,
% from the directory above. Actual implementation
% wouldn't need the import package unless the theme
% and the document are in different directories.
\usepackage{import}
\usepackage{xcolor}
\usepackage{textcomp,mathcomp}
\usepackage{cancel}
\usepackage{float}
\usepackage{forloop}


\usepackage{mhchem}
\usepackage{multido}

\definecolor{yorhabg}{HTML}{C8C2AA}
\definecolor{yorhafg}{HTML}{4D493E}
\definecolor{yorhagrid}{HTML}{B5AF9C}
\definecolor{gruvred}{HTML}{91221D}

\pagecolor{yorhabg}
\color{yorhafg}

% \import{/home/bhuv/Documents/latex docs/math/}{preamble.sty}

% Removes padding above title
\usepackage{titling}
\setlength{\droptitle}{-10em}

% Font package
\usepackage[T1]{fontenc}

\usepackage{fouriernc}

\usepackage{calc}


\usepackage{sectsty}
\usepackage{graphicx}
\usepackage{amsmath}
\usepackage{amssymb}
\usepackage[skins]{tcolorbox}

\usepackage{tikz}
\usepackage{eso-pic}
\usetikzlibrary{calc,shadows.blur}

\AddToShipoutPictureBG{%
\begin{tikzpicture}[remember picture, overlay,
                    help lines/.append style={line width=0.05pt, color=yorhagrid}]
  \draw[help lines] (current page.south west) grid[step=5pt]
                    (current page.north east);
\end{tikzpicture}%
}

% Margins
\topmargin=0in
\evensidemargin=0in
\oddsidemargin=0in
\textwidth=6.5in
\textheight=9.0in
\headsep=0.25in

\AtBeginEnvironment{tcolorbox}{\small}

\newtcolorbox{question}{%
    enhanced,
    colback=yorhabg,
    colframe=yorhafg,
    coltext=yorhafg,
    coltitle=yorhabg,
    title=\textbf{Question},
    arc=0pt,
    outer arc=0pt,
    drop shadow southeast,
    sharp corners
}

\newtcolorbox{boxx}{%
    enhanced,
    colback=yorhabg,
    colframe=yorhafg,
    coltext=yorhafg,
    coltitle=yorhabg,
    arc=0pt,
    outer arc=0pt,
    drop shadow southeast,
    sharp corners
}

\newtcolorbox{imp}{enhanced,arc=0mm,colback=yorhabg,colframe=gruvred,leftrule=10mm,coltext=yorhafg,%
overlay={\node[anchor=west,outer sep=2pt] at (frame.west) {\includegraphics[width=6mm]{images/image.png}}; }}

\newcommand\bb[1]{\textcolor{yorhafg}{\textbf{#1}}}

\newcommand\ph{\textrm{pH}}
\newcommand\poh{\textrm{pOH}}

\newcommand\mybox[2][]{\tikz[overlay]\node[fill=yorhafg, color=yorhabg, inner sep=2pt, anchor=text, rectangle, rounded corners=1mm,#1] {#2};\phantom{#2}}

\title{\textbf{P-Block Elements}}
\author{ Blxke }
\date{\today}

\begin{document}
\maketitle    


\section*{Group 13 (Boron family)}

\begin{itemize}
    \item Boron (B) : [He] \(2s^2 \ 2p^1\)
    \item Aluminium (Al) : [Ne] \(3s^2\ 3p^1\ 3d^0\)
    \item Gallium (Ga) : [Ar] \(3d^{10}\ 4s^2\ 4p^1\ 4d^0 \)
    \item Indium (In) : [Kr] \(4d^{10}\ 5s^2\ 5p^1\ 5d^0 \)
    \item Thallium (Tl) : [Xe] \(4f^{14}\ 5d^{10}\  6s^2\ 6p^1\ 6d^0\)       
\end{itemize}

\subsection*{Physical Properties.}

\subsubsection*{Atomic radii.}

\begin{center}
    \begin{boxx}
        \begin{tikzpicture}
            
            \node[] at (2, 0) {\underline{Atomic radii}};
            \node[draw] at (5, 0) {units = pm};    

                \draw[line width=10px] (0, -1.0) -- (2.3426, -1.0) node[right]{0.885};
            \draw[line width=10px] (0, -2.0) -- (3.7853, -2.0) node[right]{1.43};
            \draw[line width=10px] (0, -3.0) -- (3.2426, -3.0) node[right]{1.225};
            \draw[line width=10px] (0, -4.0) -- (4.4206, -4.0) node[right]{1.67};
            \draw[line width=10px] (0, -5.0) -- (4.5, -5.0) node[right]{1.7};


                \node at (-0.5, -1.0) {Br};
            \node at (-0.5, -2.0) {Al};
            \node at (-0.5, -3.0) {Ga};
            \node at (-0.5, -4.0) {In};
            \node at (-0.5, -5.0) {Tl};

            \draw[gruvred] (2.0426, -1.0) -- (3.4853, -2.0) -- (2.9426, -3.0) -- (4.1206000000000005, -4.0) -- (4.2, -5.0);

                \fill[yorhabg] (2.0426 , -1.0) circle (1pt);
            \draw[yorhabg] (2.0426 , -1.0) circle (2pt);
            \fill[yorhabg] (3.4853 , -2.0) circle (1pt);
            \draw[yorhabg] (3.4853 , -2.0) circle (2pt);
            \fill[yorhabg] (2.9426 , -3.0) circle (1pt);
            \draw[yorhabg] (2.9426 , -3.0) circle (2pt);
            \fill[yorhabg] (4.1206000000000005 , -4.0) circle (1pt);
            \draw[yorhabg] (4.1206000000000005 , -4.0) circle (2pt);
            \fill[yorhabg] (4.2 , -5.0) circle (1pt);
            \draw[yorhabg] (4.2 , -5.0) circle (2pt);
            
        \end{tikzpicture}
    \end{boxx}
\end{center}

The atomic radius of gallium is lesser than aluminium due to the poor shielding effect of the d-orbital in the penultimate shell. 

\vspace*{0.1in}

\hrule

\subsubsection*{Density.}

\begin{center}
    \begin{boxx}
        \begin{tikzpicture}
            \node[] at (2, 0) {\underline{Density}};
            \node[draw] at (5, 0) {units = g/cm\(^3\)};    

                \draw[line width=10px] (0, -1.0) -- (0.8924, -1.0) node[right]{2.34};
            \draw[line width=10px] (0, -2.0) -- (1.0297, -2.0) node[right]{2.7};
            \draw[line width=10px] (0, -3.0) -- (2.2538, -3.0) node[right]{5.91};
            \draw[line width=10px] (0, -4.0) -- (2.7877, -4.0) node[right]{7.31};
            \draw[line width=10px] (0, -5.0) -- (4.5, -5.0) node[right]{11.8};


                \node at (-0.5, -1.0) {Br};
            \node at (-0.5, -2.0) {Al};
            \node at (-0.5, -3.0) {Ga};
            \node at (-0.5, -4.0) {In};
            \node at (-0.5, -5.0) {Tl};

            \draw[gruvred] (0.5924, -1.0) -- (0.7297, -2.0) -- (1.9538, -3.0) -- (2.4877000000000002, -4.0) -- (4.2, -5.0);

                \fill[yorhabg] (0.5924 , -1.0) circle (1pt);
            \draw[yorhabg] (0.5924 , -1.0) circle (2pt);
            \fill[yorhabg] (0.7297 , -2.0) circle (1pt);
            \draw[yorhabg] (0.7297 , -2.0) circle (2pt);
            \fill[yorhabg] (1.9538 , -3.0) circle (1pt);
            \draw[yorhabg] (1.9538 , -3.0) circle (2pt);
            \fill[yorhabg] (2.4877000000000002 , -4.0) circle (1pt);
            \draw[yorhabg] (2.4877000000000002 , -4.0) circle (2pt);
            \fill[yorhabg] (4.2 , -5.0) circle (1pt);
            \draw[yorhabg] (4.2 , -5.0) circle (2pt);
            

        \end{tikzpicture}
    \end{boxx}
\end{center}

The density of Boron and Aluminium is low because of their low atomic masses compared to Gallium, Indium and Thallium.

\vspace*{0.1in} 

\hrule

\subsubsection*{Melting \& Boiling Points.}

\begin{center}
    \begin{boxx}
        \begin{tikzpicture}
            \node[] at (2, 0) {\underline{Melting Point}};
            \node[draw] at (5, 0) {units = \(^o\)C};    

                \draw[line width=10px] (0, -1.0) -- (4.5, -1.0) node[right]{2075};
            \draw[line width=10px] (0, -2.0) -- (1.4313, -2.0) node[right]{660};
            \draw[line width=10px] (0, -3.0) -- (0.0644, -3.0) node[right]{29.7};
            \draw[line width=10px] (0, -4.0) -- (0.3396, -4.0) node[right]{156.6};
            \draw[line width=10px] (0, -5.0) -- (0.6593, -5.0) node[right]{304};


                \node at (-0.5, -1.0) {Br};
            \node at (-0.5, -2.0) {Al};
            \node at (-0.5, -3.0) {Ga};
            \node at (-0.5, -4.0) {In};
            \node at (-0.5, -5.0) {Tl};


        \end{tikzpicture}
    \end{boxx}
\end{center}

\begin{itemize}
    \item The high melting point of Boron is high due to the fact that it exists as a giant covalent polymer in both solid and liquid state. 
    \item Gallium has an unusual structure, leading to a low melting point. 
    \item Other elements (Al, In, Tl) have a Close Packed Metal structure. 
\end{itemize}

\vspace*{0.1in} 

\begin{center}
    \begin{boxx}
        \begin{tikzpicture}
            \node[] at (2, 0) {\underline{Boiling Point}};
            \node[draw] at (5, 0) {units = \(^o\)C};    

                \draw[line width=10px] (0, -1.0) -- (4.5, -1.0) node[right]{4000};
            \draw[line width=10px] (0, -2.0) -- (2.8339, -2.0) node[right]{2519};
            \draw[line width=10px] (0, -3.0) -- (2.4795, -3.0) node[right]{2204};
            \draw[line width=10px] (0, -4.0) -- (2.331, -4.0) node[right]{2072};
            \draw[line width=10px] (0, -5.0) -- (1.6166, -5.0) node[right]{1437};


                \node at (-0.5, -1.0) {Br};
            \node at (-0.5, -2.0) {Al};
            \node at (-0.5, -3.0) {Ga};
            \node at (-0.5, -4.0) {In};
            \node at (-0.5, -5.0) {Tl};

            \draw[gruvred] (4.2, -1.0) -- (2.5339, -2.0) -- (2.1795, -3.0) -- (2.031, -4.0) -- (1.3166, -5.0);

                \fill[yorhabg] (4.2 , -1.0) circle (1pt);
            \draw[yorhabg] (4.2 , -1.0) circle (2pt);
            \fill[yorhabg] (2.5339 , -2.0) circle (1pt);
            \draw[yorhabg] (2.5339 , -2.0) circle (2pt);
            \fill[yorhabg] (2.1795 , -3.0) circle (1pt);
            \draw[yorhabg] (2.1795 , -3.0) circle (2pt);
            \fill[yorhabg] (2.031 , -4.0) circle (1pt);
            \draw[yorhabg] (2.031 , -4.0) circle (2pt);
            \fill[yorhabg] (1.3166 , -5.0) circle (1pt);
            \draw[yorhabg] (1.3166 , -5.0) circle (2pt);
            

        \end{tikzpicture}
    \end{boxx}
\end{center}

This shows that the strength of the intermolecular forces in the liquid state of the boron family decreases down the group. 

\vspace*{0.1in} 
\hrule

\newpage

\subsubsection*{Ionisation Enthalpy.}

\begin{center}
    \begin{boxx}
        \begin{tikzpicture}
            \node[] at (2, 0) {\underline{Ionisation Enthalpy}};
            \node[draw] at (5, 0) {units = kJ/mol};    

                \draw[line width=10px] (0, -1.0) -- (4.5, -1.0) node[right]{801};
            \draw[line width=10px] (0, -2.0) -- (3.2416, -2.0) node[right]{577};
            \draw[line width=10px] (0, -3.0) -- (3.2528, -3.0) node[right]{579};
            \draw[line width=10px] (0, -4.0) -- (3.1348, -4.0) node[right]{558};
            \draw[line width=10px] (0, -5.0) -- (3.309, -5.0) node[right]{589};


                \node at (-0.5, -1.0) {Br};
            \node at (-0.5, -2.0) {Al};
            \node at (-0.5, -3.0) {Ga};
            \node at (-0.5, -4.0) {In};
            \node at (-0.5, -5.0) {Tl};


            \draw[gruvred] (4.2, -1.0) -- (2.9416, -2.0) -- (2.9528000000000003, -3.0) -- (2.8348, -4.0) -- (3.0090000000000003, -5.0);

                \fill[yorhabg] (4.2 , -1.0) circle (1pt);
            \draw[yorhabg] (4.2 , -1.0) circle (2pt);
            \fill[yorhabg] (2.9416 , -2.0) circle (1pt);
            \draw[yorhabg] (2.9416 , -2.0) circle (2pt);
            \fill[yorhabg] (2.9528000000000003 , -3.0) circle (1pt);
            \draw[yorhabg] (2.9528000000000003 , -3.0) circle (2pt);
            \fill[yorhabg] (2.8348 , -4.0) circle (1pt);
            \draw[yorhabg] (2.8348 , -4.0) circle (2pt);
            \fill[yorhabg] (3.0090000000000003 , -5.0) circle (1pt);
            \draw[yorhabg] (3.0090000000000003 , -5.0) circle (2pt);

        \end{tikzpicture}
    \end{boxx}
\end{center}

The inconsistent trend is due to the poor shielding effect of \(d\) and \(f\) orbitals. 

\vspace*{0.1in} 

In case of Ga, the shielding effect leads to an increase in I.E from aluminium. 

In case of Thallium (Tl), the presence of \(f\) orbital leads to increased I.E from indium. 

\vspace*{0.1in} 

\hrule

\begin{center}
    \begin{boxx}
        \begin{tikzpicture}
            \node[] at (1, 0) {\underline{Electronegativity}};
            \node[draw] at (4.5, 0) {units = Pauling Scale};    

                \draw[line width=10px] (0, -1.0) -- (4.5, -1.0) node[right]{2.0};
            \draw[line width=10px] (0, -2.0) -- (3.6, -2.0) node[right]{1.6};
            \draw[line width=10px] (0, -3.0) -- (4.05, -3.0) node[right]{1.8};
            \draw[line width=10px] (0, -4.0) -- (4.05, -4.0) node[right]{1.8};
            \draw[line width=10px] (0, -5.0) -- (4.05, -5.0) node[right]{1.8};


                \node at (-0.5, -1.0) {Br};
            \node at (-0.5, -2.0) {Al};
            \node at (-0.5, -3.0) {Ga};
            \node at (-0.5, -4.0) {In};
            \node at (-0.5, -5.0) {Tl};


                \fill[yorhabg] (4.2 , -1.0) circle (1pt);
            \draw[yorhabg] (4.2 , -1.0) circle (2pt);
            \fill[yorhabg] (3.3000000000000003 , -2.0) circle (1pt);
            \draw[yorhabg] (3.3000000000000003 , -2.0) circle (2pt);
            \fill[yorhabg] (3.75 , -3.0) circle (1pt);
            \draw[yorhabg] (3.75 , -3.0) circle (2pt);
            \fill[yorhabg] (3.75 , -4.0) circle (1pt);
            \draw[yorhabg] (3.75 , -4.0) circle (2pt);
            \fill[yorhabg] (3.75 , -5.0) circle (1pt);
            \draw[yorhabg] (3.75 , -5.0) circle (2pt);
            \draw[gruvred] (4.2, -1.0) -- (3.3000000000000003, -2.0) -- (3.75, -3.0) -- (3.75, -4.0) -- (3.75, -5.0);
        \end{tikzpicture}
    \end{boxx}
\end{center}

This is due to the \textit{discrepancies in the atomic radii}. 

\vspace*{0.1in} 
\hrule 

\subsubsection*{Oxidation state.}

Other than boron, all other elements exhibit \textbf{+1}  and \textbf{+3} oxidation states whereas boron exhibits only +3 oxidation state.
\vspace*{0.1in} 

The stability of +3 oxidation state \textbf{decreases down the group} due to inert pair effect. 

\vspace*{0.1in}
\hrule

\subsection*{Chemical Properties.}

\subsubsection*{Allotropy.}

\begin{itemize}
    \item Boron is the only element exhibiting allotropy in its group. 
    \item It exists in both crystalline and amorphous form.
    \item It is unreactive in the crystalline form. 
\end{itemize}

\subsubsection*{Reactivity towards Air.}

\textbf{1. Boron.} 
\vspace*{0.1in} 

Amorphous boron, on heating in presence of air, reacts with oxygen and forms \textbf{Boron trioxide} (\ce{ B2O3 }).

\[
    \ce{ 4B(s) + 3O2(g) ->[\Delta] 2B2O3(s) } 
\]

At high temperatures, it reacts with nitrogen and forms \textbf{Nitrides}. 

\[
    \ce{ 2B(s) + N2(g) ->[High\ Temperature] 2BN(s) } 
\]

\vspace*{0.2in} 

\textbf{2. Aluminium.} 
\vspace*{0.1in} 

Aluminium usually doesn't react with dry air. But, it forms a thin oxide layer on its surface \textbf{when reacted with moist air}.  

\[
    \ce{ 4Al + 3O2 ->[Moist Air] 2Al2O3 } 
\]

It forms nitrides at high temperatures as well.  

\[
    \ce{  2Al + N2 ->[high temperature] 2AlN}
\]


\vspace*{0.2in} 

\textbf{3. Gallium \& Indium.} \(\implies\) These both are not affected by air. 

\vspace*{0.2in} 

\textbf{4. Thallium.}  \(\implies\) It forms an oxide layer on its surface and hence it is \textbf{preserved in oil.}  

\vspace*{0.1in} 

\hrule

\subsubsection*{Reactivity towards water.}

\textbf{1. Boron.} 

Boron is unaffected by air or water. But reacts with \textbf{red-hot steam}. 

\[
    \ce{ 2B + \underset{ (steam) }{ \ce{ 3H2O }  } -> B2O3 + 3H2 } 
\]

\vspace*{0.1in} 

\textbf{2. Aluminium.}

Aluminium decomposes cold water if there is no oxide layer present. 

\[
    \ce{ \underset{ (No\ oxide\ layer) }{ 2Al} + 3H2O -> Al2O3 + H2 } 
\]

\vspace*{0.1in} 

\textbf{3. Gallium \& Indium.} \(\implies\) are not affected by cold or hot water, unless oxygen is present. 

\vspace*{0.1in} 

\textbf{4. Thallium.} forms a hydroxide in moist air. 

\[
    \ce{ 4Tl + 2H2O + O2 -> 4TlOH } 
\]

\vspace*{0.2in}

\hrule

\subsubsection*{Reacting with acids \& alkalis.}

\textbf{1. Boron}. 

Boron doesn't react with acids or alkalis even at high temperatures. 

\vspace*{0.1in} 

\noindent 
\textbf{2. Aluminium.}

With an acid,
\[
    \ce{ 2Al(s) + 6HCl (aq) -> 2Al (aq) + 6Cl (aq) + 3H2(g) }
\]

With a base
\[
    \ce{  2Al + 6NaOH + 6H2O -> 2Na+ + [Al(OH)4]- + 3H2}
\]

Thus, aluminium shows amphoteric nature. 

\vspace*{0.1in} 
\hrule

\subsubsection*{Reactivity towards Halogens.}

The Elements of group-13 react with halides and exhibit \textbf{trivalency} in these cases. 

\begin{imp}
    Boron's halides are covalent whereas the halides of other elements are ionic. The covalent nature is due to the small size \& high electronegativity of Boron.
\end{imp}

Moreover, the halides of boron DOES NOT dimerize whereas the rest of the elements in the group form dimers either via \textit{hydrogen or coordination bonds}. 

\vspace*{0.1in} 
\hrule




\end{document}

